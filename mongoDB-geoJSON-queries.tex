% Options for packages loaded elsewhere
\PassOptionsToPackage{unicode}{hyperref}
\PassOptionsToPackage{hyphens}{url}
%
\documentclass[
]{article}
\usepackage{amsmath,amssymb}
\usepackage{lmodern}
\usepackage{iftex}
\ifPDFTeX
  \usepackage[T1]{fontenc}
  \usepackage[utf8]{inputenc}
  \usepackage{textcomp} % provide euro and other symbols
\else % if luatex or xetex
  \usepackage{unicode-math}
  \defaultfontfeatures{Scale=MatchLowercase}
  \defaultfontfeatures[\rmfamily]{Ligatures=TeX,Scale=1}
\fi
% Use upquote if available, for straight quotes in verbatim environments
\IfFileExists{upquote.sty}{\usepackage{upquote}}{}
\IfFileExists{microtype.sty}{% use microtype if available
  \usepackage[]{microtype}
  \UseMicrotypeSet[protrusion]{basicmath} % disable protrusion for tt fonts
}{}
\makeatletter
\@ifundefined{KOMAClassName}{% if non-KOMA class
  \IfFileExists{parskip.sty}{%
    \usepackage{parskip}
  }{% else
    \setlength{\parindent}{0pt}
    \setlength{\parskip}{6pt plus 2pt minus 1pt}}
}{% if KOMA class
  \KOMAoptions{parskip=half}}
\makeatother
\usepackage{xcolor}
\usepackage[margin=1in]{geometry}
\usepackage{graphicx}
\makeatletter
\def\maxwidth{\ifdim\Gin@nat@width>\linewidth\linewidth\else\Gin@nat@width\fi}
\def\maxheight{\ifdim\Gin@nat@height>\textheight\textheight\else\Gin@nat@height\fi}
\makeatother
% Scale images if necessary, so that they will not overflow the page
% margins by default, and it is still possible to overwrite the defaults
% using explicit options in \includegraphics[width, height, ...]{}
\setkeys{Gin}{width=\maxwidth,height=\maxheight,keepaspectratio}
% Set default figure placement to htbp
\makeatletter
\def\fps@figure{htbp}
\makeatother
\setlength{\emergencystretch}{3em} % prevent overfull lines
\providecommand{\tightlist}{%
  \setlength{\itemsep}{0pt}\setlength{\parskip}{0pt}}
\setcounter{secnumdepth}{-\maxdimen} % remove section numbering
\ifLuaTeX
  \usepackage{selnolig}  % disable illegal ligatures
\fi
\IfFileExists{bookmark.sty}{\usepackage{bookmark}}{\usepackage{hyperref}}
\IfFileExists{xurl.sty}{\usepackage{xurl}}{} % add URL line breaks if available
\urlstyle{same} % disable monospaced font for URLs
\hypersetup{
  pdftitle={mongoDB GeoJSON queries},
  pdfauthor={Ac Hybl},
  hidelinks,
  pdfcreator={LaTeX via pandoc}}

\title{mongoDB GeoJSON queries}
\author{Ac Hybl}
\date{2022-10-26}

\begin{document}
\maketitle

\hypertarget{data-import}{%
\section{Data Import}\label{data-import}}

first we import data either through the GUI or using this mongosh
command

\begin{verbatim}
# general form
mongoimport <path to file> -c=collectionName
# I just used the GUI 
\end{verbatim}

\begin{figure}
\centering
\includegraphics{images/paste-52E3FF14.png}
\caption{\includegraphics{images/paste-102CACA6.png}}
\end{figure}

\includegraphics{images/paste-0F1F2E0C.png}

\hypertarget{geo-spacial-data-structuring}{%
\section{Geo-spacial data
structuring}\label{geo-spacial-data-structuring}}

When I look at my earthquake documents after the import, we can see that
there are \texttt{latitude} and \texttt{longitude} fields but not and
GeoJSON structures such as \texttt{geometry} or \texttt{Point}. To do
get this, we have to create a new field using mongoDB commands:

\begin{verbatim}
# what I had to do:
db.earthquake.updateMany({}, [{$set: {location: {type: "Point", coordinates: ["$longitude", "$latitude"]}}}]);



# break that down into a general use case ⬇
db.collectionName.updateMany(
  # first parameter is filter document:
  {},
  # I enter an enmpty document if I want to
  # keep all documents in collection
  
  
  # next, provide a list of updates:
  [
    {
      $set: # update command (set|addfield|unset|etc.)
      {
        fieldName:
        # the value is next. it can be of any supported type
        "a value"
      }
    }
  ]
# close function call
);



# my use case
db.collectionName.updateMany(
  {},
  [
    {
      $set: 
      {
        location: 
        {
          type: "Point", 
          coordinates: ["$longitude", "$latitude"]
        }
      }
    }
  ]
);
\end{verbatim}

After this, we have a location object for every earthquake document.

\begin{verbatim}
"location": {
    "type": "Point",
    "coordinates": [
      -117.901,
      36.168
    ]
  }
\end{verbatim}

\end{document}
